\documentclass[a4paper,11pt,oneside]{memoir}

\usepackage{TUINFDA}

\thesistitle{Proposal}
%\thesissubtitle{KNX For Safety Critical Environments} % optional
\thesissubtitle{Highly available KNX networks}
\thesisdate{25.8.2014}

% all titles and designations have to be gender-related!
\thesisdegree{Diplom-Ingenieurin}{Diplom-Ingenieurin}
\thesiscurriculum{Technische Inforumatik}{Computer Engineering} % your study
\thesisverfassung{Verfasserin} % Verfasser
\thesisauthor{Harald Glanzer} % your name
\thesisauthoraddress{Hardtgasse 25/12A, 1190 Wien} % your address
\thesismatrikelno{0727156} % your registration number

\thesisbetreins{Ao.Univ.Prof.Dr. Wolfgang Kastner}
\thesisbetrzwei{Dipl. Ing. Lukas Krammer}

% define page numbering styles
\makepagestyle{numberCorner}


%\makeevenfoot{numberCorner}{\thepage}{}{}
%\makeevenfoot{numberCorner}{}{}{\thepage}
%\makeoddfoot{numberCorner}{}{}{\thepage}

% define custom macros for specific formats or names
\newcommand{\uml}[1]{\texttt{#1}}
\newcommand{\cd}{\textsf{Class Diagram}}

\usepackage[backend=bibtex, sorting=none]{biblatex}
\addbibresource{references.bib}


%\RequirePackage[backend=bibtex]{biblatex}

\begin{document}

\captionnamefont{\bfseries}

%%%%%%%%%%%%%%%%%%%%%%%%%%%%%%%%%%%%%%%%%
%%%   FRONTMATTER    %%%%%%%%%%%%%%%%%%%%
%%%%%%%%%%%%%%%%%%%%%%%%%%%%%%%%%%%%%%%%%
\frontmatter
\pagenumbering{normal}

%%%%%%%%%%%%%%%%%%%%%%%%%%%%%%%%%%%%%%%%%
%%%   TITLEPAGES    %%%%%%%%%%%%%%%%%%%%%
%%%%%%%%%%%%%%%%%%%%%%%%%%%%%%%%%%%%%%%%%

% an english translation may follow
% $Id: titlepage.tex 1752 2010-03-20 11:07:02Z tkren $
%
% TU Wien - Faculty of Informatics
% thesis titlepage
%
% This titlepage is using the geometry package, see
% <http://www.ctan.org/macros/latex/contrib/geometry/geometry.pdf>
%
% For questions and comments send an email to
% Thomas Krennwallner <tkren@kr.tuwien.ac.at>
% or to Petra Brosch <brosch@big.tuwien.ac.at>
%

\selectlanguage{english}

% setup page dimensions for titlepage
\newgeometry{left=2.4cm,right=2.4cm,bottom=2.5cm,top=2cm}

% force baselineskip and parindent
%\newlength{\tmpbaselineskip}
%\setlength{\tmpbaselineskip}{\baselineskip}
%\setlength{\baselineskip}{13.6pt}
%\newlength{\tmpparindent}
%\setlength{\tmpparindent}{\parindent}
%\setlength{\parindent}{17pt}

% first titlepage
\thispagestyle{tuinftitlepage}

\pagenumbering{Roman}

\begin{center}
{\ \vspace{5cm}}

\begin{minipage}[t][2.8cm][s]{\textwidth}%
\centering
\thesistitlefonthuge\sffamily\bfseries\tuinfthesistitle\\
\bigskip
\bigskip
\bigskip
\bigskip
{\thesistitlefontHUGE\sffamily\bfseries\tuinfthesissubtitle\HUGE}
\\
~
\\
%Building a Security Gateway for KNX networks with improved availability
\end{minipage}

{\ \vspace{6cm}}

\vspace{0pt}\raggedright\sffamily\Large
\begin{minipage}[t][4cm][t]{\textwidth}%
  \begin{tabbing}%
	     \hspace{40mm} \= \hspace{66mm} \kill
	    Author:\> B.Sc. Harald Glanzer\\  
	    \hspace{40mm} \= \hspace{66mm} \kill
	    Advisor:\> \tuinfthesisbetreins\\
	    Assistance: \> \tuinfthesisbetrzwei\\
	    \\
	    Department:\>  Institute of Computer Aided Automation\\
	    \>Automation Systems Group
     \end{tabbing}
\end{minipage}

{\ \vspace{4cm}}


\begin{minipage}[t][1.5cm][t]{\textwidth}%
  \vspace{0pt}\sffamily\thesistitlefontnormalsize
  \begin{tabbing}%
    \hspace{45mm} \= \hspace{63mm} \= \hspace{51mm} \kill
    Vienna, \tuinfthesisdate \> {\raggedright\rule{51mm}{0.5pt}} \> {\raggedright\rule{51mm}{0.5pt}} \\
    \> \begin{minipage}[t][0.5cm][t]{51mm}\centering (Signature of Author)\end{minipage}
    \> \begin{minipage}[t][0.5cm][t]{51mm}\centering (Signature of Advisor)\end{minipage}
    \end{tabbing}
\end{minipage}

\end{center} % optional


%%%%%%%%%%%%%%%%%%%%%%%%%%%%%%%%%%%%%%%%%
%%%   ABSTARCT    %%%%%%%%%%%%%%%%%%%%%%%
%%%%%%%%%%%%%%%%%%%%%%%%%%%%%%%%%%%%%%%%%

\section{Problem statement}

KNX is an open communications protocol for Building Automation (BA).
It uses a layered structure and supports wired communication over twisted pair
and power line as well wireless communication by radio transmission. 
Additionally, it allows communication with TCP/IP hosts by special gateways. 
As such, it can be used for controlling traditional services like Heating, Ventilation and Air Conditioning (HVAC), but also for more
sophisticated applications 
like surveillance or fire alarm systems of buildings \cite{knxapps}.
\\
Given these potential applications, a variety of attacks would be possible, ranging from DoS attacks to disable fire alarm systems 
by simply physically shortcutting a line
connection, to opening doors by intercepting and replaying datagrams.
%Such attacks can be conducted by passive advisories by replay attacks or eavesdropping, interception, altering and injecting of arbitrary telegrams.

The countermeasure, providing integrity, confidentiality and authenticity, consists of authentication
between the sender and receiver of a message, and encryption of these messages, combined in a security scheme
called Authenticated Encryption (AE).

%To gain confidentiality, block or stream ciphers are used to garble the cleartext message into a block respectively stream
%of pseudo-random binary zeros and ones. Without possessing the key, the encryption must not
%be reversible in feasible time to protect the secret message from unauthorized entities.
%Integrity, on the other side, is achieved by taking blocks of input data and producing a short (compared
%to the data itself, which can consist of an arbitrarily large number of blocks) Message Authentication Code (MAC)
%for this data, which will be sent along with the data. The MAC function is some kind
%of one-way-function and must be 'difficult' to reverse, i.e. find parts or the whole block 
%which was feed to the function to obtain the given MAC.
%This way, altering of the sent data can be
%detected and the corresponding datagram can be discarded. Analog to encryption,
%it must be infeasible to find other MACs for the same message.   
Availability, in general, can only be achieved by redundancy, i.e. by using replicated resources. Therefore, all
resources needed for transmitting data between two points must exist redundantly and independently from 
each other.
\\

The basic KNX Standard is regarded as insufficient because the standard simply doesn't provide any encryption, authentication 
or availability mechanisms \cite{knxSpec}. To address the cryptographic issues, extensions are available, but no solution is disposable for 
improved availability with integrated security mechanisms.

%This work's overall topic is to propose a solution for KNX networks with a higher level of availability, compared to standard KNX networks.
%To allow the deployment in safety critical environments, cryptographic measurements to provide confidentiality and integrity will be integrated as well.

\section{Expected results}

The overall goal of this work is to develop a concept for a secure and high-available KNX network that also considers interoperability and compatibility, 
allowing the usage in environments with increased safety-critical requirements.
To achieve this, so called security gateways will be used. These gateways will possess 2 kind of KNX interfaces: one kind of interface will be
connected to standard, unsecured KNX networks.
The second kind of interface constitutes the entry point to a secured KNX network which is connected to the secure interfaces of other
security gateways. To achieve higher availability, these secure interfaces as well as the communication lines must exist redundantly.

To show the feasibility of the solution by a proof of concept, a demonstration network  shall be built.
For the security gateways, RaspberryPis in combination with KNX-USB-dongles will be used. Therefore, the RasperryPis
are acting as gateways between the secure and the insecure KNX networks, each of them running a master daemon responsible
for reading datagrams from the KNX insecure world, encrypting and authenticating them and sending them over the secure
KNX lines.

%\\
%It will be shown that booth concepts must go hand in hand to achieve confidentiality. 
%Consequently, the security layer will use strong encryption \textit{and} authentication to guard
%against attacks.
%\\

%To gain availability, a handshake protocol will be used, as well as some kind of 'heartbeat' mechanism to
%detect outages of one communication line as soon as possible, allowing to switch to the replicated line 
%instantaneously

%For gaining availability, it would be possible to define some kind of 'heartbeat' protocol which checks the status
%of the 2 lines, and switches the line in case of a failure. A disadvantage of this approach is that a considerable
%amount of bus load is generated. Alternatively, it is possible to just copy every KNX message to booth lines
%(after encrypting it), and use some kind of counter to remove duplicates on the receiving side. This solution
%needs no additional traffic, and is also 'faster' because no fail over recognition time is needed. 

%While it would be possible to encrypt and authenticate only the actual payloads, and leave the status
%datagrams unencrypted, a cleaner and more elegant solution is to encrypt and authenticate booth message 
%types. This makes it possible to use the concept of \textit{divide et impera} in an uncompromising way.
%Additionally, attacks against the heartbeat - mechanism become impossible.

It is important to note that the practical part of this work will only
handle the twisted-pair media of KNX, although the basic principles can be deployed in a modified manner in
wireless and power-line networks as well.

A threat analysis will be conducted to prove that the system can withstand the defined attacks and is robust,
i.e. that it can recover from erroneous states. This will be done by exposing the demonstration network to the various defined attacks.  

\section{Methodological approach}

Every secure system will just work within some defined barriers - it is impossible to build a system that
is secure under all circumstances. So, the very first step will be to define a realistic
thread scenario by studying typical attacks against Building Automation Systems (BAS) \cite{granzer2006security}.
Ensuring security in a network is a complex and comprehensive topic, fortunately, canonical ways how to employ authenticated
encryption exist. 
Therefore, the next step is to research state-of-the-art techniques to decide which ciphers should be used, how the keys will
be  distributed, what kind of MAC to use and how cleartext messages will be mapped to encrypted ones, and vice versa.
Also, a reasonable concept has to be found how
to guarantee high availability, considering the limited resources of standard KNX devices and the limited bandwidth of KNX TP-1.

To separate security and availability related functions, distinct layers for this two tasks will be used.

To implement the real-world testing environment, a master daemon will be written, consisting of a stack composed of EIBD,
the security / key exchange layer in combination with a cryptographic library and the availability layer. The master daemon will be written
in C, and a GNU/Linux based operating system named Raspbian will be used for the RaspberryPis. EIBD is a C++ daemon for communication
with different KNX backends through an API. 

After implementation, it is evaluated whether the approach withstands the defined attacks,
and whether the protocol works in practice.
 

% \section{Structure of the work}
% \begin{enumerate}
%   \item Introduction
%   \item KNX
%   \begin{itemize}
%     \item Specification
%     \item Overview of the defined layers
%     \item Attacks on KNX
%   \end{itemize}
%  
%   \item State of the art
%       \begin{itemize}
%   
%   \item Computer Security
%   \begin{itemize}
%     \item Definition of Security
%     \item Types of Attacks
%     \item Basic Concepts of Number Theory used(Fermat, Euler)
%     \item Basic Concepts of Probability Theory used
%     \item Encryption Schemes
%     \item Authentication Schemes
%     \item Authenticated Encryption
%     \item Distribution of Keys
%     \item Symmetric vs. Asymmetric Encryption
%     \item Random Number Generators
%   \end{itemize}
% 
%   \item Availability
%   \begin{itemize}
%     \item Requirements for Availability
%     \item Protocol Used
%     \item Resynchronization
%   \end{itemize}
%     \end{itemize}
% 
%   \item Security Concept
%   \begin{itemize}
%     \item Key Generation and Distribution
%     \item Definition of the Status Messages
%     \item Definition of the MAC
%     \item Definition of the Cipher
%     \item Definition of the Tunneling Mode
%     
%     \item Evaluation
%   \begin{itemize}
%     \item Implemenation 
%     \item Performance
%     \item Threat analysis
%   \end{itemize}
%   \end{itemize}
%   
%   
%  
%   \item Critical Reflections
% 
% \end{enumerate}
  
\section{State of the art}

%While these 2 approaches try to bring confidentiality to KNX, this work will also add availability to KNX, as stated 
%above. Additionally, this work will not rely on key servers, thus eliminating the need for a single 'point of trust'.
%Last but not least, the work of Salvatore Cavalieri and Giovanni Cutuli contains a serious design flaw by delegating 
%the task of key generation to a so called 'controller' with a fixed KNX address, which will be used to negotiate a key,
%thus using the KNX address as authentication property. If it is possible to deploy a malicious knx device into the network
%which responds 'faster' than the controller, the whole architecture is rendered useless. \cite{knxsecExt} by these authors
%addresses this shortcoming, but again, just makes theoretical assumptions.

The rapid growth of electronic data processing and digital communication enforces the need for secure and available systems.
Information security, consisting of the triad confidentiality, integrity and availability, tries to achieve such systems. 
Cryptography uses ciphers to achieve integrity and confidentiality and is also a prerequisite for availability. To improve the latter one,
replication is used. A replicated service uses redundant components, providing multiple outcomes. A voter mechanism is used to determine
which outcome is used.

A fundamental property of ciphers is the data format, i.e. if the data is to be handled in form of blocks or in form of a continuous
data stream. Stream ciphers can be provable "perfect secret" in principle
and can be implemented as simple as bitwise xor'ing the key and the data (i.e., the one-time pad). 

Block ciphers come in 2 flavors: pseudo-random functions (PRF), and pseudo-random permutations (PRP). While the latter one is reversible,
this is not true for the first one. Therefor, PRFs can only be used in constructions which do not depend on a reverse function, for
example like "Feistel Networks". A widely used encryption standard is "AES", the advanced encryption standard, derived from a block cipher
called "Rijndel".
This construction is reversible (i.e. is a PRP) and is also called a "substitution-permutation-network", named after it's 2 basic building blocks.
 
 Another very fundamental difference is which kind of keys are used, i.e. symmetric vs. asymmetric encryption: while symmetric encryption is superior in
 regards of performance, symmetric keys need an already established, secure channel for key exchange. Because of
 that, a mixture of booth modes is used in prominent protocols (i.e., TLS \cite{rfc2246}).
 The key exchange algorithm defines how the keys get distributed to all devices which are part of the secure network.
 It would be possible to fix a key, place this key on all devices and use this Pre Shared Key (PSK) for symmetric
 encryption, but this key cannot be changed at a later point without direct interaction on all devices. A better way is to use some
 kind of asymmetric key which is used to establish session keys, which can be used in a symmetric manner. Well known examples
 of asymmetric or public key algorithms are "RSA" \cite{Rivest:1978:MOD:359340.359342} and "Diffie-Helmann" \cite{1055638}. While the latter one was originally based on exponentiation,
 a new method based on elliptic curves has been found which can achieve the same level of security with shorter keys.
 
 Another important distinguishing point is the mode of operation: this basically means which construct is used to transform cleartext data
 into the needed pseudo-random representation, which underlying cipher is used, and also defines how this transformation is reversed for decryption.
 There are modes for encryption and authentication, some modes can also be used for booth tasks (i.e. cipher block chaining, CBC). 
 
 Additionally, this property decides what
 is done first: encryption, and afterwards authentication of the encrypted data, or authentication of the cleartext data,
 appending the obtained MAC to the cleartext data, and encrypting data and MAC afterwards. Depending on this ordering and
 what modes are used for authentication and encryption, this option may enable attacks like padding oracles \cite{Vaudenay02securityflaws}. Another possibility
 is to generate a MAC for the cleartext data, and only encrypt the data itself. Obviously, care must be taken that the MAC does not
 carry any information about the cleartext data.
 
Open source, high-level APIs like OpenSSL or Crypto++ offer a wide range of authentication, encryption and key exchange modes.
These libraries are widely used and actively maintained, so there is no need to reimplement these primitives.

Depending on the mode of operation and the length of the used keys, there exists an upper bound on
 how many messages can be sent securely without changing the key. Therefore, it must be either made sure that this number 
 cannot be achieved in a reasonable time, or some kind of key-renegotiation has to be used, which is the task of the key management algorithm.
 Nevertheless, if a session key and the corresponding traffic gets known to an adversary, all the past data will be disclosed (and all future traffic
 if no new key is used). Protocols like Off The Record Messaging (OTR) \cite{Borisov:2004:OCW:1029179.1029200}, avoid this problem by using short term session keys, and thus providing
 Perfect Forward Secrecy (PFS). This property ensures that, even if a key is known to an adversary, no future and no past messages can be decrypted (beside of
 one single message). Problematic about OTR is its lack of support for multi-party conversations, a feature that is tried to be achieved with
 Multi Party OTR (mpOTR) \cite{Goldberg:2009:MOM:1653662.1653705,}\cite{Liu:2013:IGO:2517840.2517867}.
% Finally, a critical prerequisite is the used Pseudo Random Number Generator (PRNG).
% While the used platform contains a hardware based PRNG, is has to be examined if the provided entropy can be considered
% secure, because a predictable PRNG turns all other cryptographic measurements useless.
\\
IPv4 \cite{rfc791}, which is until today one of the basic building blocks for worldwide internet communication, suffers a serious limitation because it
does not offer confidentiality and integrity on the network level. The problem was mitigated in 2 ways: one solution was 
the introduction of another layer - TLS -  above the IPv4 layer, responsible for handling security.
The second way was the design of the IPsec \cite{rfc4301} 
extension, which authenticates and encrypts data sent with IPv4 by defining two security services,
namely the Authentication Header (AH) to provide authenticity, and the Encapsulating Security Payload (ESP) for confidentiality. Internet
Key Exchange (IKE), is used as key negotiating protocol. This way,
IPsec can provide end-to-end encryption and protect the payload of higher level protocols like TCP or UDP.

As stated earlier, KNX defines no methods for securing datagrams in the original proposal - a situation comparable to the origin IPv4 standard.
For KNX, the following extensions which compensate this flaw exist:

KNX Application Note 157 specifies an optional security layer for KNX networks \cite{knx_data_sec}, while 
KNX Application Note 158 improves security for KNX/IP networks \cite{knx_ip_sec}.
EIBsec uses key servers to provide secure management and group communication and implements a proof of
concept \cite{KraInnosec2013}. Salvatore Cavalieri and Giovanni Cutuli propose another way how to authenticate and encrypt KNX traffic \cite{knxsec}.

\section{Relatedness to Computer Engineering}

Modern cryptography relies heavily on number theory and probabilistic theory and is the basis of this work.
The practical work will be to implement the multi-threaded daemons on the RaspberryPis, written in the low-level programming language
C, by using the C++ API offered by EIBD.

Related lectures:

\begin{itemize}
 \item 104.271 VO Discrete Mathematics 
 \item 104.272 UE Discrete Mathematics 
 \item 184.189 VU Cryptography 
 \item 182.721 VO Embedded Systems Engineering 
 \item 182.722 LU Embedded Systems Engineering 
 %\item 389.152 VO Network Security 
 \item 389.166 VU Signal Processing 1
 \item 183.624 VU Home and Building Automation 

\end{itemize}

\printbibliography

\end{document}
