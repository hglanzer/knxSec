\documentclass[a4paper,12pt,twoside]{memoir}
\usepackage{TUINFDA}



\thesistitle{Proposal}
\thesissubtitle{KNX For Safety Critical Environments} % optional
\thesisdate{25.8.2014}

% all titles and designations have to be gender-related!
\thesisdegree{Diplom-Ingenieurin}{Diplom-Ingenieurin}
\thesiscurriculum{Wirtschaftsinformatik}{Business Informatics} % your study
\thesisverfassung{Verfasserin} % Verfasser
\thesisauthor{Martina Musterfrau} % your name
\thesisauthoraddress{Musterplatz 1, 1111 Wien} % your address
\thesismatrikelno{0123456} % your registration number

\thesisbetreins{Ao.Univ.Prof.Dr. Wolfgang Kastner}
\thesisbetrzwei{Dipl. Ing. Lukas Krammer}

% define page numbering styles
\makepagestyle{numberCorner}
\makeevenfoot{numberCorner}{\thepage}{}{}
\makeoddfoot{numberCorner}{}{}{\thepage}

% define custom macros for specific formats or names
\newcommand{\uml}[1]{\texttt{#1}}
\newcommand{\cd}{\textsf{Class Diagram}}

\begin{document}

\captionnamefont{\bfseries}

%%%%%%%%%%%%%%%%%%%%%%%%%%%%%%%%%%%%%%%%%
%%%   FRONTMATTER    %%%%%%%%%%%%%%%%%%%%
%%%%%%%%%%%%%%%%%%%%%%%%%%%%%%%%%%%%%%%%%
\frontmatter
\pagenumbering{normal}

%%%%%%%%%%%%%%%%%%%%%%%%%%%%%%%%%%%%%%%%%
%%%   TITLEPAGES    %%%%%%%%%%%%%%%%%%%%%
%%%%%%%%%%%%%%%%%%%%%%%%%%%%%%%%%%%%%%%%%

% an english translation may follow
% $Id: titlepage.tex 1752 2010-03-20 11:07:02Z tkren $
%
% TU Wien - Faculty of Informatics
% thesis titlepage
%
% This titlepage is using the geometry package, see
% <http://www.ctan.org/macros/latex/contrib/geometry/geometry.pdf>
%
% For questions and comments send an email to
% Thomas Krennwallner <tkren@kr.tuwien.ac.at>
% or to Petra Brosch <brosch@big.tuwien.ac.at>
%

\selectlanguage{english}

% setup page dimensions for titlepage
\newgeometry{left=2.4cm,right=2.4cm,bottom=2.5cm,top=2cm}

% force baselineskip and parindent
%\newlength{\tmpbaselineskip}
%\setlength{\tmpbaselineskip}{\baselineskip}
%\setlength{\baselineskip}{13.6pt}
%\newlength{\tmpparindent}
%\setlength{\tmpparindent}{\parindent}
%\setlength{\parindent}{17pt}

% first titlepage
\thispagestyle{tuinftitlepage}

\pagenumbering{Roman}

\begin{center}
{\ \vspace{5cm}}

\begin{minipage}[t][2.8cm][s]{\textwidth}%
\centering
\thesistitlefonthuge\sffamily\bfseries\tuinfthesistitle\\
\bigskip
\bigskip
\bigskip
\bigskip
{\thesistitlefontHUGE\sffamily\bfseries\tuinfthesissubtitle\HUGE}
\\
~
\\
%Building a Security Gateway for KNX networks with improved availability
\end{minipage}

{\ \vspace{6cm}}

\vspace{0pt}\raggedright\sffamily\Large
\begin{minipage}[t][4cm][t]{\textwidth}%
  \begin{tabbing}%
	     \hspace{40mm} \= \hspace{66mm} \kill
	    Author:\> B.Sc. Harald Glanzer\\  
	    \hspace{40mm} \= \hspace{66mm} \kill
	    Advisor:\> \tuinfthesisbetreins\\
	    Assistance: \> \tuinfthesisbetrzwei\\
	    \\
	    Department:\>  Institute of Computer Aided Automation\\
	    \>Automation Systems Group
     \end{tabbing}
\end{minipage}

{\ \vspace{4cm}}


\begin{minipage}[t][1.5cm][t]{\textwidth}%
  \vspace{0pt}\sffamily\thesistitlefontnormalsize
  \begin{tabbing}%
    \hspace{45mm} \= \hspace{63mm} \= \hspace{51mm} \kill
    Vienna, \tuinfthesisdate \> {\raggedright\rule{51mm}{0.5pt}} \> {\raggedright\rule{51mm}{0.5pt}} \\
    \> \begin{minipage}[t][0.5cm][t]{51mm}\centering (Signature of Author)\end{minipage}
    \> \begin{minipage}[t][0.5cm][t]{51mm}\centering (Signature of Advisor)\end{minipage}
    \end{tabbing}
\end{minipage}

\end{center} % optional


%%%%%%%%%%%%%%%%%%%%%%%%%%%%%%%%%%%%%%%%%
%%%   ABSTARCT    %%%%%%%%%%%%%%%%%%%%%%%
%%%%%%%%%%%%%%%%%%%%%%%%%%%%%%%%%%%%%%%%%

\section{Motivation and problem statement}

KNX is an open communications protocol for industrial building automation.
It uses a layered structure and supports wired communication over twisted pair
and power line as well wireless communication by radio or infrared transmission. 
Additionally, it defines gateways to the TCP/IP world. 
As such, it can be used for remotely controlling traditional services like HVAC, shortcut
for heating, ventilation and air conditioning, but also for more sophisticated applications\cite{knxapps}
like surveillance or fire alarm systems of buildings - ranging in size
from private homes to huge office buildings.
\\
Given these potential applications, such a communications protocol would be a weak point an adversary
could exploit if there are no countermeasures deployed.
Possible attacks range from DOS attacks by simply physically shortcutting a twisted line
connection to replay attacks or eavesdropping, interception, altering and injecting of arbitrary telegrams.
One countermeasure providing integrity, confidentiality and authenticity consists of authentication
between sender and receiver of a message, and encryption of these messages, combined in a security scheme
called 'Authenticated Encryption'.
\\
Encryption uses block or stream ciphers to garble the cleartext message into a line of pseudo-random binary
zeros and ones, unreadable for anyone not knowing the decryption key.
\\
Authenticity, on the other side, takes the input data and produces a short 'MAC'
(shortcut for 'message authentication code')
for this data, which will be sent along with the encrypted data. This way, tampering of the sent data can be
detected and the corresponding datagram will be discarded.
\\
Availability, in general, can only be achieved by redundancy, by using replicated resources. Therefore, all
resources needed for transmitting data between two points must exists redundant and independent from 
each other.
\\
\\
This work's overall topic is to provide an extension for KNX which adds an dependability layer, allowing 
to use KNX in a security-critical environment. As will be shown, the security concept defined in the
actual KNX Standard must be regarded as insufficient because the methods for encryption used can be attacked
easily.
\\
It is important to note that this work will only
handle the twisted-pair variant of KNX, although the authenticated encryption methods can be deployed in
wireless networks as well.


\section{Expected results}

The final goal of this work is to build a prototype of a secure and dependable KNX network. This network
will consist of 2 raspberry pi single board computers, connected to each other with 2 KNX-usb-dongles, 
constituting the dependable section of the KNX network. Additionally, each raspberry pi is connected by
another usb dongle to another KNX  network in the standard, unsecured way. Therefore, the 2 rasperry pi's
are gateways between a secure and an unsecured KNX network, each of them running a master daemon responsible
for reading datagrams from the KNX 'unsecured' world, encrypting them and sending them over the secured
KNX lines. On the counterpart, the message will be decrypted and checked for integrity and sent
on its further way if the message is found to be untampered. Additionally the daemon monitors the state of the 2
replicated KNX channels.
\\
While it is not obvious that confidentiality cannot be achieved without authenticity and vice versa,
it will be shown that one concept without its counterpart is useless. 
Consequently, the dependability layer will use strong encryption \textit{and} authentication to guard
against attacks.
\\
To gain availability, a handshake protocol will be used, as well as some kind of 'heartbeat' mechanism to
detect outages of one communication line as soon as possible, allowing to switch to the replicated line 
instantaneously	.
\\
While it would be possible to encrypt and authenticate only the actual payloads, and leave the status
datagrams unencrypted, a cleaner and more elegant solution is to encrypt and authenticate booth message 
types. This makes it possible to use the concept of \textit{divide et impera} in an uncompromising way.
Additionally, attacks against the heartbeat - mechanism become impossible.

\section{Methodological approach}

The methodological approach will be broken down into these steps:

\begin{itemize} 
 \item Setup of the working environment
 \\
 As stated, rasperry pi's will be used. These are small but powerful single board computers, based on ARM processors, 
 using a Linux kernel as operating system. A C++ KNX API named 'eibd' exists and will be used for
 the KNX/USB interface. When this first step is completed, a gateway from KNX to usb is available which
 provides a data link layer to be used by other layers.
 
 \item Literature Review
 \\
 Avoiding eavesdropping of and tampering with messages in a network is a complex and comprehensive
 topic. Fortunately, canonical ways how to employ authenticated
 encryption exists, therefore the first step is to decide which ciphers should be used, how the keys will
 be  distributed and what kind of MAC to use.
 \item Design of the security layer
 \\
 This step will use the results of the literature review to properly implement the security concept found. 
 
 \item Design of the dependability layer and heartbeat protocol
 \\
 The dependability layer will be built upon the security layer, consisting of the handshake protocol and an additional
 heartbeat protocol, signaling faulty conditions to the master daemon.
\end{itemize}

\section{Structure of the work}
\begin{enumerate}
 \item Introduction
 \item KNX
 \begin{itemize}
  \item Specification
  \item Security Concept used
  \item Attacks on KNX Encryption
 \end{itemize}
 
 \item Computer Security
 \begin{itemize}
 \item Definition of Security
  \item Types of Attacks
  \item Basic Concepts of Number Theory used(Fermat, Euler)
  \item Basic Concepts of Propability Theory used
  \item Encryption Schemes
  \item Authentication Schemes
  \item Authenticated Encryption
  \item Distribution of Keys
  \item Symmetric vs. Asymmetric Encryption
  \item Actual Concept used
 \end{itemize}
 \item Availability
 \begin{itemize}
  \item Handshake Protocol Used
  \item Heartbeat Mechanism Used
  \item Types of Datagrams
 \end{itemize}
 \item Codelistings

\end{enumerate}
  
\section{State of the art}

As stated earlier, KNX already defines a method for securing datagrams, but unfortunately this method is
to be considered insecure and can be attacked easily.
On the other hand, some proposals which promise to fix this issue exist, nevertheless these projects didn't
make it further than any theoretical level:

\begin{itemize}
 \item 'eibsec'\cite{eibsec} defines a security extension to KNX which uses AES encryption and dedicated key
servers
 \item In \cite{knxsec}Salvatore Cavalieri and Giovanni Cutuli propose another way how to authenticate and
 encrypt KNX traffic.
\end{itemize}

Instead of self-implementing the needed crypto-primitives, the use of a API like 'crypto++' or ' Keyczar'
is favored because these libraries are widely used, open source, maintained and have proven to be secure
in the field. 

\section{Relatedness to Computer Engineering}

Modern cryptography relies heavily on number theory and probabilistic theory and is the basis of this work. 
To correctly implement this cryptosystem, the kernel on the target platform will be interfaced by low level
programming constructs, thus combining two major topics of the academic program.

Related lectures:

\begin{itemize}
 \item 104.271 VO Discrete Mathematics 
 \item 104.272 UE Discrete Mathematics 
 \item 184.189 VU Cryptography 
 \item 182.721 VO Embedded Systems Engineering 
 \item 182.722 LU Embedded Systems Engineering 
 \item 389.152 VO Network Security 
 \item 389.166 VU Signal Processing 1
 \item 183.624 VU Home and Building Automation 

\end{itemize}

\begin{thebibliography}{9}

 \bibitem{knxapps}
  KNX Association
  \emph{\LaTeX: Introducing Security and Authentication in KNX }.
  http://www.knx.org/fileadmin/downloads/08\%20-\%20KNX\%20Flyers/KNX\%20Solutions/KNX\_Solutions\_English.pdf

\bibitem{eibsec}
  Granzer, Kastner, Neugschwandtner
  \emph{\LaTeX:EIBSEC: A Security Extension to KNX}.
  http://www.knx.org/fileadmin/downloads/05\%20-\%20KNX\%20Partners/03\%20-\%20Becoming\%20a\%20KNX\%20Scientific\%20Partner/2006-11\%20Scientific\%20Conference\%20Papers\%20Vienna/05\_granzer-eibsec\_security-knxsci06-website.pdf
  2006.

  \bibitem{knxsec}
  Salvatore Cavalieri, Giovanni Cutuli
  \emph{\LaTeX: Introducing Security and Authentication in KNX }.
  http://www.knx.org/fileadmin/downloads/05\%20-\%20KNX\%20Partners/03\%20-\%20Becoming\%20a\%20KNX\%20Scientific\%20Partner/2008-11\%20Conference/presentations/session2.pdf  
  
\end{thebibliography}
\end{document}
