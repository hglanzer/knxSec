\section{Security}

The basic building stones of information security are confidentiality, integrity and availability, also called the \textit{CIA - triad} \cite{stallings}.
\textbf{Confidentiality} is used to protect sensitive information from eavesdroppers who are not allowed to get knowledge of that information. 
\textbf{Integrity} ensures that some kind of information can not be altered by third-parties, or that such a modification can be detected by the
receiver of the information, and also includes information non-repudiation.
\textbf{Availability} ensures that information, which is needed by an entity to provide some service, is accessible.
All three properties go hand-in-hand with each other because a successful attack on one property may allow attacking another one. For example, if a
confidentiality attack against a computer system responsible for money transfers can be conducted to steal a password used for controlling this system
, an attacker
can subsequently render the system unusable, therefore compromising availability. On the other hand, the attacker could also try to remain undetected and change
booking orders, thus mounting an attack against the integrity of the system. Thus, these 3 basic concepts are interleaved, and building a system which honors
only parts of them will most likely lead to an insecure system.

Additionally to the CIA - triad, sometimes two more concepts are used in the security field: \textbf{Authenticity} is tied to integrity and ensures the property
of being genuine, while \textbf{Accountability} allows to link actions performed on a system uniquely to the entity responsible for them.
