\section{Motivation}

\gls{knx} is an open communications protocol for \gls{hbas}.
It uses a layered structure and supports wired communication over twisted pair
and power line as well as wireless communication by radio transmission. 
Additionally, it supports communication with \gls{tcp}/\gls{ip} hosts by special gateways. 
As such, it can be used for controlling traditional services like \gls{hvac}, but also for more
sophisticated applications like surveillance or fire alarm systems of buildings \cite{knxapps}.
\\
\\
Driven by the need to reduce maintenance costs and to improve usability, the application of \glspl{hbas} is no longer limited to traditional \gls{hvac} services.
Modern building management includes much more different and more sophisticated tasks like elevator control, alarm systems or access control, to name a few.
To reduce costs it would be natural to bring together these services under the control of one management system, a claim also supported through improved 
standardization efforts.
\\
Given these potential applications, a wide range of attacks would be possible. 
Replay attacks by intercepting and replaying datagrams would allow an adversary to introduce arbitrary \gls{knx} traffic, switching doors
or disabling burglar alarms. Passive attackers can monitor the bus traffic to analyze the types of \gls{knx} devices within the network, gathering knowledge that can be used
to develop further attack strategies.
\gls{dos} attacks, disabling all directly connected \gls{knx} devices, can be conducted by simply physically shortcutting or interrupting a line
connection, rendering the corresponding network segment unavailable. Clearly, such attacks must be precluded for sensitive services like fire or burglar alarms,
relying on the availability of the communication network.
\\
High availability, in general, can only be achieved by redundancy, i.e. by using replicated resources. Therefore, all
resources needed for transmitting data between two points must exist redundantly and independently from each other.
\\
The countermeasure against eavesdropping and replay attacks, providing integrity, confidentiality and authenticity, consists of authentication
between the sender and receiver of a message, and encryption of these messages, combined in a security scheme called \gls{ae}.
%In contrast to traditional \gls{hvac} tasks, where security concerns were neglected, integration of these additional services calls for application of measurements
%to prevent misuse and improved availability. Avoiding misused is in general achieved by integrating confidentiality and authenticity measurements, a fact which
%was neglected by the \gls{knx} consortium at first, but was fixed in several extensions afterwards.
%Improved availability, which is of course of prime interest in applications like burglar alarm systems or elevator control, can be achieved in various ways. 
\section{Problem statement}
The origin \gls{knx} standard did not include countermeasures to prevent adversary misuse. This deficit was fixed afterwards by various extensions, bringing confidentiality and
authenticity to \gls{knx}. Nevertheless, up to today no methods for increasing availability are provided, a fact considered problematic in view of the potential application domains.
\\
For example, a burglar can render an alarm system useless if he can shortcut the bus lines, thus suppressing the (encrypted) alarm messages. In contrast to such an malicious attack, a transient
hardware failure can disable a system too, unacceptable for systems handling fire alarms or controlling elevators.

\section{Aim of the work}

The overall goal of this work is to develop a concept for a secure and highly available \gls{knx} network that also considers interoperability and compatibility, 
allowing the usage in environments even with increased safety-critical requirements. The proposed solution shall be resistant against malicious adversary as well as against
transient hardware failures.
\\
To achieve this, so called security gateways will be used. These gateways will possess two kinds of \gls{knx} interfaces: one kind of interface will be
connected to standard, unsecured \gls{knx} networks.
The second interface constitutes the entry point to a secured \gls{knx} network which is connected to the secure interfaces of other
security gateways. To achieve higher availability, these secure interfaces and the respective communication lines must exist redundantly. This ensures that
even in case of a \gls{dos} attack, communication within the segment is possible.
\\
\\
To show the feasibility of the solution by a proof of concept, a demonstration network shall be built.
For the security gateways, RaspberryPis in combination with \gls{knx}-\gls{usb}-dongles will be used. Therefore, the RasperryPis
are acting as gateways between the secure and the insecure \gls{knx} networks, each of them running a master daemon responsible
for reading datagrams from the \gls{knx} insecure world, encrypting and authenticating them and sending them over the secure
\gls{knx} lines.
\\
It is important to note that the practical part of this work will only
handle the twisted-pair media of \gls{knx} (i.e. \gls{knx} \gls{TP}-1), although the basic principles can be deployed in a modified manner in
wireless and power line networks as well.
\\
A threat analysis will be conducted to prove that the system can withstand the defined attacks and is robust,
i.e. that it can recover from erroneous states. This will be done by exposing the demonstration network to the various defined attacks. 

\section{Structure of the work}

Chapter \ref{ch:prerequisites} introduces the required prerequisites for providing integrity and confidentiality, handling symmetric and asymmetric ciphers. 
Chapter \ref{ch:availability} explains the term availability and is concluded by state-of-the art technologies implementing highly-available communication networks.
\\
Chapter \ref{ch:knx} explains the structure of the \gls{knx} standard, and introduces the most important cryptographic extensions.
\\
Chapter \ref{ch:concept} suggests a \gls{knx} prototype, applicable in environments with increased availability needs.
\\
Finally, Chapter \ref{ch:implementation} explains the implementation of the prototype, evaluates it and discusses the results.



