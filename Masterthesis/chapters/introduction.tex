\section{Motivation}

\gls{knx} is an open communications protocol for \gls{hbas}.
It uses a layered structure and supports wired communication over twisted pair
and power line as well as wireless communication by radio transmission. 
Additionally, it supports communication with \gls{ip} hosts by special routers. 
As such, it can be used for controlling traditional services like \gls{hvac}, but also for more
sophisticated applications like surveillance or fire alarm systems of buildings \cite{knxapps}.
\\
\\
Driven by the need to reduce maintenance costs and to improve usability, the application of \glspl{hbas} is no longer limited to traditional \gls{hvac} services.
Modern building management includes much more different and more sophisticated tasks like elevator control, alarm systems or access control, to name just a few.
To reduce costs it would be natural to bring together these services under the control of one management system, a claim also supported through improved 
standardization efforts.
\\
Given these potential applications, a wide range of attacks would be possible. 
Replay attacks by intercepting and replaying datagrams would allow an adversary to introduce arbitrary \gls{knx} traffic, switching doors
or disabling burglar alarms. Passive attackers can monitor the bus traffic to analyze the types of \gls{knx} devices within the network, gathering knowledge that can be used
to develop further attack strategies.
\gls{dos} attacks, disabling all directly connected \gls{knx} devices, can be conducted by simply physically shortcutting or interrupting a line
connection, rendering the corresponding network segment unavailable. Clearly, such attacks must be precluded for sensitive services like fire or burglar alarm systems,
relying on the availability of the communication network.
\\
High availability, in general, can only be achieved by redundancy, i.e., by using replicated resources. Therefore, all
resources needed for transmitting data between two points must exist redundantly and independently from each other.
\\
The countermeasure against eavesdropping and replay attacks, providing integrity, confidentiality and authenticity, consists of authentication
between the sender and receiver of a message, and encryption of these messages, combined in a security scheme called \gls{ae}.
%In contrast to traditional \gls{hvac} tasks, where security concerns were neglected, integration of these additional services calls for application of measurements
%to prevent misuse and improved availability. Avoiding misused is in general achieved by integrating confidentiality and authenticity measurements, a fact which
%was neglected by the \gls{knx} consortium at first, but was fixed in several extensions afterwards.
%Improved availability, which is of course of prime interest in applications like burglar alarm systems or elevator control, can be achieved in various ways. 
\section{Problem statement}
To get insights into the deficits of \gls{knx}, the special requirements of \gls{hbas} are introduced. Afterwards, the shortcomings of \gls{knx}, avoiding its
deployment in critical environments, are explained. In Section \ref{sec:knxSecConcept}, these shortcomings and the extensions dealing with them are reviewed in more depth.

\subsection{Security in HBAS}\label{hbaSec}
\glspl{hbas} emerged from automation systems, originally used for \gls{hvac} building control.
Central building management, leading to \textit{intelligent buildings}, promises
reduced maintenance costs, energy savings and improved user comfort \cite{1435745}, compensating the initially higher investment costs of such buildings.
Following these arguments, it would be natural to integrate additional building management functions like alarm systems, access control or communication systems,
exploiting already existing infrastructure like cabling and thus benefiting from synergy effects.
\\
This trend was contradicted by the fact that in the early days of \gls{hbas}, communication security was not considered a critical requirement.
Firstly, the communication was done over wires,
i.e., physical access to the network would have been necessary for attacking the network \cite{knxSpec}. Secondly, the possible threats by misusing \gls{hvac} applications
were considered negligible. Additionally, the devices used in such networks were characterized by very limited processing power - thus, the comprehensive
use of cryptographic measurements would have put remarkable computing loads onto these devices and was therefore considered impracticable.
\\
These arguments turn out to be true only at first glance. Because \gls{hbas} are operational over years, excluding short-time, temporary physical access is often impossible
for wired networks and nearly impossible for
wireless networks. Regarding the second argument, it is easy to see that simple acts of vandalism, for example shutting down the lighting system of a company building, can result
in considerable financial losses.
\\
\\
Today, the necessary processing power is available even on embedded systems, meanwhile systems integration continued until a point where security concerns could
no longer be neglected. It follows that such an \gls{hbas} must be protected against misuse on all existing levels. 
Communication networks for \gls{hbas} are usually built upon a two-tier model, consisting of a field- an a backbone level. The field level contains \gls{sac},
interacting with the environment and performing the control functions. They are interconnected by the backbone network. Here, \glspl{md}, used for configuration,
visualization and monitoring, as well as \glspl{icd}, connecting physical segments, are found. Special \glspl{icd} may act as gateways, providing a connection to foreign networks. 
\\
\\
Due to this topology, two different classes of attacks are possible: network- and device attacks \cite{5332331}.

\subsubsection{Network attacks}
Here, the adversary either tries to compromise the field or backbone network. In the first scenario, the attacker can analyze or modify the control application. In the latter he
gains access to the concentrated network data and can thus obtain a global view of the system.
\\
To protect a communication network it would be possible to use security mechanism known from the \gls{ip} world. 
Unfortunately, these mechanisms often cannot be mapped directly to \glspl{hbas} because of the introduced overhead. Using
established techniques like \gls{vpn}, \gls{tls} or \gls{ipsec} is therefore reserved for the backbone level where \gls{ip} is used.
\\
At the field level, \gls{ip} is hardly ever used, a fact that excludes \gls{ip} based technologies here.

\subsubsection{Device attacks}
Alternatively, the three different kinds of devices can be aimed at by either attacking a \gls{sac} to manipulate its behavior or by attacking a \gls{icd} to access data of the segment.
Finally, the attacker can launch an attack against the \glspl{md} to gain management access to the other two types of devices.
\\
Such device attacks are divided into three categories: software attacks, side-channel and physical attacks, extensively surveyed in \cite{5332331} and \cite{secAn}.
\\
\\
Being a communication protocol, \gls{knx} can not handle attacks against devices.
Nevertheless, the origin standard did not include countermeasures to prevent network attacks. This deficit was fixed afterwards by various extensions, bringing confidentiality and
authenticity to \gls{knx}. Nevertheless, up to today no methods for increasing availability are provided, a fact considered problematic in view of the potential application domains.
\\
For example, a burglar can render an alarm system useless if he can shortcut the bus lines, thus suppressing the (encrypted) alarm messages. In contrast to such a malicious attack, a transient
hardware failure can also disable a system, unacceptable for systems handling fire alarms or controlling elevators. 

\section{Aim of the work}

The overall goal of this work is to develop a concept for a secure and highly available \gls{knx} network that also considers interoperability and compatibility, 
allowing the usage in environments even with increased safety-critical requirements. The proposed solution shall be resistant against malicious adversaries, as well as against
transient hardware failures.
\\
To achieve this, so called security gateways will be used. These gateways will possess two kinds of \gls{knx} interfaces: one kind of interface will be
connected to standard, unsecured \gls{knx} networks.
The second interface constitutes the entry point to a secured \gls{knx} network which is connected to the secure interfaces of other
security gateways. To achieve higher availability, these secure interfaces and the respective communication lines must exist redundantly. This ensures that
even in case of a \gls{dos} attack, communication within the segment is possible.
\\
\\
To show the feasibility of the solution by a proof of concept, a demonstration network shall be built.
For the security gateways, RaspberryPis in combination with \gls{knx}-\gls{usb}-dongles will be used. Therefore, the RasperryPis
are acting as gateways between the secure and the insecure \gls{knx} networks, each of them running a master daemon responsible
for reading datagrams from the \gls{knx} insecure world, encrypting and authenticating them and sending them over the secure
\gls{knx} lines.
\\
It is important to note that the practical part of this work will only
handle the twisted-pair media of \gls{knx} (i.e., \gls{knx} \gls{TP}-1), although the basic principles can be deployed in a modified manner in
wireless and power line networks as well.
\\
A threat analysis will be conducted to prove that the system can withstand the defined attacks and is robust,
i.e., that it can recover from erroneous states. This will be done by exposing the demonstration network to the various defined attacks. 

\section{Structure of the work}

Chapter \ref{ch:prerequisites} introduces the required prerequisites for providing integrity and confidentiality, handling symmetric and asymmetric ciphers. 
Afterwards Chapter \ref{ch:availability} explains the term availability and is concluded by state-of-the art technologies implementing highly-available communication networks.
\\
The structure of the \gls{knx} standard and the most important cryptographic extensions are explained in Chapter \ref{ch:knx}.   
\\
Chapter \ref{ch:concept} suggests a \gls{knx} extension that is applicable in environments with increased availability needs.
\\
The work is concluded by Chapter \ref{ch:implementation}, which explains the implementation of the prototype, evaluates it and discusses the results.



