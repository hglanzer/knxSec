\chapter*{Kurzfassung}
\gls{hga} befasst sich mit Systemen zur Steuerung bzw. Regelung von Gebäudeanwendungen, wie Heizungen, Klimaanlagen oder Beleuchtungssystemen. Raumparameter
können so zentral gesteuert werden, wodurch sowohl der Verwaltungsaufwand als auch den Energieverbrauch gesenkt und der Komfort erhöht werden kann. 
Sicherheitstechnische Überlegungen spielten traditionellerweise eine untergeordnete Rolle.
Einerseits standen die Ressourcen auf den verwendeten Plattformen oft nicht zur Verfügung, andererseits wurden die Auswirkungen eines böswilligen Eingriffes in das System vernachlässigt.
%Zusätzlich ist z.B. die optimale Regelung der Raumtemperatur in einem
%Bürogebäude wichtig für den Komfort der darin arbeitenden Menschen, was sich auch auf die Produktivität auswirken kann.
\\
Firmengebäude wie Privatgebäude beherbergen jedoch eine viel grössere Anzahl an Applikationen, man denke hier an Zugangskontrollen, 
Alarmanlagen oder Brandlöschsysteme. Diese Gruppe von Anwendungen stellt höhere Anforderungen an das zugrunde liegende technische System:
Türe dürfen nur von authorisierten Personen geöffnet werden, Alarmanlagen dürfen sich nicht einfach von Einbrechern deaktivieren lassen, und Brandmeldesysteme müssen im
Extremfall mit einem hohen Grad an Zuverlässigkeit funktionieren. 
\\
Diese unterschiedlichen Anforderungen führten zu einem Auseinanderwachsen der vorhandenen Systeme. Das Zusammenführen 
von kritischen und unkritischen Systemen würde einerseits den Verwaltungsaufwand weiter senken und zusätzlich erlauben, die vorhandene Infrakstruktur, z.B. die Verkabelung,
für beide Anwendungsgebiete zu verwenden.
\\
\\
Diese Arbeit beschäftigt sich deshalb mit einer Erweiterung des \gls{knx} Standards für die \gls{hga}, die auch in kritischen Umgebungen eingesetzt werden kann.
Dazu ist es einerseits nötig, böswillige Angriffe zu erkennen und zu verhindern als auch technische Defekte abfedern zu können. Ersteres ermöglicht der Einsatz von Kryptographie,
zweiteres kann mittels Redundanz bewerkstelligt werden. Beide Begriffe, sowie \gls{knx} selbst, werden ausführlich erläutert, gefolgt von dem erarbeitetem Lösungsvorschlag.
Der Ansatz unterteilt eine \gls{knx}-Installation in einen ungesicherten und einen gesicherten Teil. Letzterer ist geschützt gegen böswillige Angriffe und ausserdem doppelt ausgeführt,
womit ein teilweiser Ausfall kompensiert werden kann. Der Aufbau der vorgeschlagenen Lösung wird beschrieben und abschliessend der implementierte Prototyp erläutert. 
