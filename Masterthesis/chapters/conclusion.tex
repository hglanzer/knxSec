\section{Contribution}
To be able to deploy a communication system in more demanding environments, it is necessary to achieve informational security combined with mechanisms for improved availability.
For \gls{knx}, extensions for securing a network against malicious attacks exist, but these extension are not able to handle a fault concerning the communication medium, as well
as \gls{dos} attacks.
Therefore, this work proposes a way to protect against transient hardware failures, as well as active and passive adversaries. For the latter, the proposal is able to resist
restricted \gls{dos} attacks, too. This is achieved by using security gateways which are connected to each other in a redundant way. Standard \gls{knx} devices are connected 
through these gateways, which copy the client's frames into two properly secured frames and send them over both communication lines. The receiving side will check the received
frames for modification, discard one of the two copies and forward the remaining one to the destined standard \gls{knx} device.
\\
\\
It was shown that the proposed solution can withstand malicious attacks, as well as transient hardware failures of one of the secured lines. Therefore, the solution allows to connect
standard \gls{knx} devices which are spatially divided in a secure manner, bridging over areas where malicious behavior cannot be ruled out.
The proposed solution can be deployed in a 'plug-and-play' 
manner, as long as the constraints defined in Section \ref{sec:opContraints} are regarded. Thus, it is possible to add confidentiality, integrity and availability to a \gls{knx}
network just where needed, coexisting with segments with low security demands. By using more than 2 secured lines, the solution can easily be extended to use $n$ instead of 2 secure threads
and communication lines, increasing availability to an even higher level.

\section{Outlook}

The following improvements to the prototype solution are proposed: 
\begin{itemize}
 \item Using a cache for the mapping of group address to security gateways and encryption keys. A gateway receiving data from a client would send a discovery message once and
 cache the address(es) of the responsible gateway(s), together with the corresponding encryption key(s). Subsequent data transfer can be executed without the need for sending
 discovery messages first, reducing the bus load. Of course, a reasonable caching time has to be found after which a cached entry is deleted because newly connected
 clients may not be reachable within this duration.
 \item Obfuscating the size of the data service messages, which  can be achieved by adding an additional length field and corresponding padding, chosen randomly. This makes it harder
 to derive communication profiles.
 \item Attacks can be detected if frames with forged \glspl{mac2} are received, allowing to send an alert to an operator. For the used platform, this can be achieved easily by connecting
 the RaspberryPis to a restricted \gls{ip} network and sending the alarm, for example, by mail or \gls{snmp}.
 Additionally, the address of the attacking device can be added to a blacklist such that traffic coming from this address is discarded immediately.
 \end{itemize}
 
\subsubsection{Restrictions }
During development, the follow restrictions were found. Of course, the constraining assumptions made in Section \ref{sec:opContraints} are still valid. 

\begin{itemize}
  \item While \gls{eibd} can send extended frames, it does not support receiving frames with payload $\geq$ 56 byte. While this clearly violates the \gls{knx} specification, no efforts
  were made to debug this behavior.
  \item With the current \gls{eibd} \gls{api}, sending raw frames is not possible, but only application- or transport layer data units. This implies that 
  it is not possible to set the source address of a written frame. Frames delivered by a security gateway are therefore sent with the gateway's device address, instead of the 
  origin device address. To make the security gateways appear fully transparent to client devices, the \gls{api} must be extended.
 \end{itemize}


 
 