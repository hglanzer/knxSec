\section{Security in HBAS}\label{hbaSec}
\glspl{hbas} emerged from automation systems, originally used for building control, summarized as \gls{hvac}.
Central building management, leading to \textit{intelligent buildings}, promises
reduced maintenance costs, energy savings and improved user comfort \cite{1435745}, compensating the initially higher investment costs of such buildings.
Following these arguments it would be natural to integrate additional building management functions like alarm systems, access control or communication systems,
exploiting already existing infrastructure like cabling and thus benefiting from synergy effects.
\\
This trend was contradicted by the fact that in the early days of \gls{hbas}, communication security was not considered a critical requirement.
Firstly, the communication was done over wires,
i.e. physical access to the network would have been necessary for attacking the network \cite{knxSpec}. Secondly, the possible threats by misusing \gls{hvac} applications
were considered negligible. Additionally, the devices used in such networks were characterized by very limited processing power - thus, the comprehensive
use of cryptographic measurements would have put remarkable computing loads onto these devices and was therefore considered impracticable.
\\
These arguments turn out to be true only at first glance. Because \gls{hbas} are operational over years, excluding short-time, temporary physical access is often impossible
for wired networks and nearly impossible for
wireless networks. Regarding the second argument it is easy to see that simple acts of vandalism, for example shutting down the lighting system of a company building, can result
in considerable financial losses.
\\
\\
Today the necessary processing power is available even on embedded systems, meanwhile systems integration continued until a point where security concerns could
no longer be neglected. It follows that such a \gls{hbas} must be protected against misuse on all existing levels. 
Communication networks for \gls{hbas} systems are usually built upon a two-tier model, consisting of a field- an a backbone level. The field level contains \glspl{sac},
interacting with the environment and performing the control functions. They are interconnected by the backbone network. Here, \glspl{md}, used for configuration,
visualization and monitoring, as well as \glspl{icd}, connecting physical segments, are found. Special \glspl{icd} may act as gateways, providing a connection to foreign networks. 
\\
\\
Due to this topology, two different classes of attacks are possible: network- and device attacks \cite{5332331}.

\subsubsection{Network attacks}
Here, the adversary either tries to compromise the field or backbone network. In the first scenario, the attacker can analyze or modify the control application. In the latter he
gains access to the concentrated network data and can thus obtain a global view of the system.
\\
To protect a communication network it would be possible to use security mechanism known from the \gls{ip} world. 
Unfortunately, these mechanisms often cannot be mapped directly to \glspl{hbas} because of the introduced overhead. Using
established techniques like \gls{vpn}, \gls{tls} or \gls{ipsec} is therefore reserved for the backbone level where \gls{ip} is used.
\\
At the field level \gls{ip} is hardly ever used, a fact that excludes \gls{ip} based technologies here.

\subsubsection{Device attacks}
Alternatively, the three different kinds of devices can be aimed at by either attacking a \gls{sac} to manipulate it's behavior or by attacking a \gls{icd} to access data of the segment.
Finally, the attacker can launch an attack against the \glspl{md} to gain management access to the other two types of devices.
\\
Such device attacks are divided into three categories: software attacks, side-channel and physical attacks, extensively surveyed in \cite{5332331} and \cite{secAn}.