\section{Security in \gls{hbas}}\label{hbaSec}
\glspl{hbas} emerged from automation systems, originally used for building control, summarized as \gls{hvac}.
Central building management, leading to \textit{intelligent buildings}, promises
reduced maintenance costs, energy savings and improved user comfort \cite{1435745}, compensating the initially higher investment costs of such buildings.
Following these arguments it would be natural to integrate additional building management functions like alarm systems, access control or communication systems,
exploiting already existing infrastructure like cabling and thus benefiting from synergy effects.
\\
This trend was contradicted by the fact that in the early days of \gls{hbas}, communication security was not considered a critical requirement:
firstly, the communication was done over wires,
i.e. physical access to the network would have been necessary for attacking the network \cite{knxSpec}. Secondly, the possible threats by misusing \gls{hvac} applications
were considered negligible. Additionally, the devices used in such networks were characterized by very limited processing power - thus, the comprehensive
use of cryptographic measurements would have put remarkable computing loads onto these devices and was therefore considered impracticable.
\\
Today the necessary processing power is available even on embedded systems, meanwhile systems integration continued until a point where security concerns could
no longer be neglected. Therefore, the needs of \gls{hbas} regarding security is introduced next.
\\
\\
Communication networks for \gls{hbas} systems are usually built upon a two-tier model.
\\
FIXME: not finished yet.
