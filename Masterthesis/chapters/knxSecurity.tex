\section{KNX security concept}
As shown in section \ref{hbaSec}, a paradigm shift towards towards increased security in \gls{hbas} can be observed. 
Despite that fact, the basic \gls{knx} standard does not specify any security mechanisms for control information:
\\

\textit{"For KNX, security is a minor concern, as any breach of security requires local access to the network. 
In the case of KNX TP1 or KNX PL110 networks this requires even physical access to the network wires,
which in nearly all cases is impossible as the wires are inside a building or buried underground.
Hence, security aspects are less of a concern for KNX field level media."} \cite{knxSpec}
\\
\\
For \gls{knx}/\gls{ip}, the physical containment arguments do not apply. To counter this, it is proposed to use firewalls and \glspl{vpn} to prevent unauthorized access,
as well as hiding critical network parameters from public. The latter concept is also known as \textit{"security by obscurity"}, offering - if at all - only 
little protection.
\\
For management communication, a rudimentary, password-based control is used. Therefore, \gls{knx} suffers the following security flaws 
\cite{Granzer05securityin}: for management, the 
used keys are transmitted as cleartext, enabling an attacker to perform a passive attack to obtain the password. Subsequently, the attacker can mount an active
attack, injecting
arbitrary management messages. No methods are foreseen for generation or distribution of the keys.
For control information, an adversary can directly inject arbitrary messages to control the network, allowing passive and active attacks too.
These shortcomings clearly disqualified \gls{knx} for usage in critical environments, restricting its possible field of application.
\\
Today, \gls{hbas} systems are used on a large scale, and the available processing power on embedded computing platforms has risen significantly, so the deployment
of such systems would be possible also in critical environments, under the condition that proper security mechanisms are deployed. For \gls{knx}, several extensions
exist which will be introduced in the next sections.

\subsection{KNX Data Security}

In 2013, the KNX association published "Application Note 158" \cite{knx_data_sec} which specifies the \gls{knx} \gls{s-al}, providing
authentication and encryption, and the \gls{ail}, implementing access control, booth being part of the application layer.
The settlement of these functions above the transport layer allows a transparent, communication media independent end-to-end encryption.

The application layer service code 0xF31 is reserved for this purpose, indicating that a secure header and a \gls{s-al} \gls{pdu} 
follow instead of a plaintext-\gls{pdu}. This allows the flexible usage of the secure services just in situations where they are needed - otherwise, the plaintext application
layer services can be used.

The \gls{s-al} services defines modes for authenticated encryption or authentication-only of a higher-level cleartext \gls{apdu}. As underlying block cipher
\gls{aes}128 is used in \gls{ccm} mode, encrypting the payload with \gls{ctr} and providing integrity with \gls{cbc} mode. The overhead introduced by the 
\gls{mac2} is reduced by 
using only the 32 most significant bits instead of the whole 128 bit block obtained from \gls{cbc}. Source- and destination address as well as
frame- and addresstype, the \gls{tpci}, length information and a 6 byte sequence number determine the \gls{iv} for the \gls{cbc} algorithm and are therefor also protected by the \gls{mac2}.
The sequence number is a simple counter value that provides data freshness, thus preventing replay attacks, and is sent along with every \gls{s-al} \gls{pdu}.
For synchronization of this sequence number between two devices, a \gls{s-al} s	ync-service is defined. Because no sequence number can be used here to guarantee
data freshness, a challenge-response mechanism is used instead.
Two different types of keys are used: a \gls{fdsk} is used for initial setup with the \gls{ets}. The \gls{ets} then generates the \gls{tk}, which is used by the
device for securing of the outgoing messages. Consequently, every device must know the \gls{tk} of it's communication partners.

While the \gls{s-al} empowers two devices to communicate in a secure way, the \gls{ail} allows a fine-grained control which sender has access to which
data objects. Therefore, every \textit{link} (a combination of source address and data or service object) is connected with a \textit{role}, which in turn
has some specific \textit{permissions}. 

\subsection{EIBsec}

EIBsec is another extension to \gls{knx}, providing data integrity, confidentiality and freshness, allowing its deployment in security-critical environments.
A semi-centralized approach is taken here by using special key servers, responsible for dedicated sets
of keys, providing a sophisticated key management. EIBsec divides a \gls{knx} network into subnets, connected by devices called \gls{acu}. Beside their native
task, i.e. routing traffic, they
 are responsible for the key management of their network segments, which includes key generation, distribution and revocation. Every standard device that
 wishes to communicate with other devices must at first retrieve the corresponding secret key from its responsible \gls{acu}, which can therefor control the
 group membership of the requesting device by allowing or denying the request respectively by revoking the key at a later point in time.

EIBsec uses two different keys: in normal mode, a session with the keyserver is established to retrieve the session key, establishing a secure channel.
This mode uses encryption-only by utilizing a \gls{psk}, integrity must therefor be guaranteed by the sender of the message. Counter mode is used for transmitting management and
group data over the secure channel. A simple \gls{crc} is added to the payload before encryption and shall guarantee integrity. Booth modes encrypt the traffic
on transport level, allowing standard routers to handle the datagrams. As block cipher, \gls{aes}-128 is used in \gls{ctr} mode.

FIXME: zitieren / originalpaper...?

\subsection{KNX IP Security}

This work focuses on securing the \gls{knx} \gls{ip} specification, which can be used as backbone for connecting distinct \gls{knx} installations \cite{5195839}.
Thanks to the widespread use of \gls{tcp}/\gls{ip}, a wide range of physical transport mechanisms can be utilized.
\\
A structure comparable to the design of \gls{tls} is defined by building a distinct security layer, residing above the transport layer (therefore, it directly
connects the transport to the application layer, because session- and presentation layer are empty, as defined by the \gls{knx} specifications, see chapter
\ref{ch:knx}).
\\
The design distinguishes 3 different types of modes: 

in the \textit{configuration phase}, every device that wants to participate in the secure network
generates an asymmetric key pair, which is sent to the \gls{ets} over a secure channel (for example, by transmitting the data over a direct, serial connection
between the \gls{ets} host and the \gls{knx} device).
The \gls{ets}, acting as \gls{ca1}, signs the combination of \gls{ip} and public key with its own private key, thus generating a certificate, which is sent back to the device, along
with the public key of the \gls{ets}. 

After that, the \textit{key set distribution phase} starts, where a unicast and a multicast scenario are differentiated:
in the unicast case, the device initiating the communication - called client - obtains the key set from the target device by a 2-step handshake: at first, mutual entity 
authenticity is established by utilizing the certificate provided by the \gls{ets}. Afterwards the keyset is obtained from the target, which concludes the second
phase.

In the last step, secure communication can take place, i.e. the the client is able to
encrypt the data with the obtained key and sends it to the target device.

For the multicast scenario, a distinct \textit{coordinator}, responsible for maintaining the group key, is necessary. Every powered-up device
identifies the coordinator as soon as possible by broadcasting \textit{hello} requests, adopting the coordinator role if no replies are received in time.
To actually send a payload, the group key is obtained from the coordinator and the data is sent to the group, analog to the unicast case.
By adding mechanisms to detect "dead" coordinators and delegating the coordinator role to a different device, the design avoids a \gls{spof}.

