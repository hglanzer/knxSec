\section{Summary}
When examining the security mechanisms for \gls{knx} it shows up that only rudimentary precautions were defined in the basic standard simply because they were not considered
necessary. 
\\
This perspective changed and different extensions where introduced. 
Modes for application-transparent encryption and authentication were introduced with "KNX Application Note 158". Securing \gls{ip} based \gls{knx} backbones is possible by using 
"KNX IP Security".
\\
\\
Despite the \gls{knx} extensions, no solution providing high availability exists. Therefore, this work proposes a solution which brings together cryptographic measurements with
redundancy mechanisms, allowing its deployment in more demanding environments.


% Examining the \gls{knx} security measurements shows up parallels to the \gls{ipv4} world. Similar to \gls{ipv4}, only rudimentary security measurements were foreseen in the
% original standard. 
% Various \gls{knx} extensions were introduced to fix this flaw.
% For \gls{ipv4}, the problem was
% fixed in two various ways: \gls{tls} was added as a sublayer between \gls{osi} layers 4 and 5, allowing application-transparent encryption and authentication, 
% similar to KNX Application Note 158, as described above. The second solution was the introduction of \gls{ipsec}, extending \gls{ipv4} and thus enabling authentication and encryption on \gls{ip} level.