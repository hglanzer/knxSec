 

\begin{itemize}
 \item Divisiblity
 \\
$a | b \iff \exists z \in \mathbb{Z}: b = z \times a$
 \item Integer Division $r \equiv a \bmod b$ 
 \\
 $ \forall a, b \in \mathbb{Z}, b > 0:  a = z \times b + r $ with $ 0 \leq r < b$
 \item Common Divisor $c$ for $a, b \iff$
 \\
 $c | a \wedge c | b$
 \item Greatest Common Divisor(GCD) $d$:
 \\
 $d | a \wedge d | b$ and whenever $c | a \wedge c | b \rightarrow c | d$
 
 \item unlimited number of primes
 \item every number has exact 1 factorization 
\end{itemize}

\section{One Way functions}

The idea for this concept was formulated for the first time in the year 1874 by William Stanley Jevons in his book
'The Principles of Science'(page 144).
