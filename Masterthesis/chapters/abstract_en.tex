\chapter*{Abstract}
\gls{hbas} denominates systems for controlling building applications summarized under the term \gls{hvac}. Parameters can thus be controlled in a centralized manner, 
promising lower maintenance and energy costs and higher comfort. Security aspects were often neglected because the used hardware platforms lacked the needed processing power
and malicious attacks against such systems ignored.
\\
Nevertheless, corporate buildings as well as private homes contain a great variety of additional applications, for example access controls, burglar alarm or fire detection systems.
This group of applications makes much greater demands regarding the underlying technical system. Obviously, access must be only granted to authenticated persons and fire detection
systems must work reliable in case of emergency. 
\\
The different requirements lead to a separation of critical and uncritical systems, unifying them into one system would allow to further decrease maintenance costs and re-use
the existing infrastructure for both fields of application 
\\
\\
Therefore, this thesis proposes a extension to \gls{knx} which is suitable for critical environments. For this purpose it is necessary to detect and guard against malicious attacks
as well as to cope with randomly occurring hardware faults. 


DErsteres ermöglicht der Einsatz von Kryptographie,
zweiteres kann mittels Redundanz bewerkstelligt werden. Beide Begriffe sowie \gls{knx} selbst werden ausführlich erläutert, gefolgt von dem erarbeitetem Lösungsvorschlag.
Der Ansatz unterteilt eine \gls{knx}-Installation in einen ungesicherten und einen gesicherten Teil. Letzterer ist geschützt gegen böswillige Angriffe und ausserdem doppelt ausgeführt,
womit ein teilweiser Ausfall kompensiert werden kann. Abschliessend wird der implementierte Prototyp erläutert. 