\section{Master daemon}

\subsection{KNX addressing scheme}

Care must be taken that no duplicate knx addresses are used within the network. Therefore, the following addressing convention is proposed:
While it would be possible to use the same addresses on booth lines per gateway, a different scheme is used.
For the secured network, the address ranges starting at address 1.1.1 to address 1.1.15 and 1.2.1 to 1.2.15 are reserved for secure line number
1 and 2 respectively, which allows a maximum of 15 gateways. Different addresses are used mainly because it facilitates debugging. Additionally, the used
address ranges can be re-used outside the secured network by standard devices anyway.
On the unsecured lines, every gateway uses an address from the range 1.0.1 - 1.0.15. Addresses are assigned in a linearly ascending way, so gateway number 1
uses addresses 1.1.1 and 1.2.1 for secure lines 1 and 2, and 1.0.1 for its unsecured line.

\section{Evaluation}

\subsection{Synchronization stage}
Packages in the discovery stage are not encrypted, allowing a passive adversary to learn the value of the global counter value $Ctr_{global}$. Nevertheless,
this counter is only used to avoid deterministic encryption (see \ref{deterministicEnc}) and is of no use for the attacker.
\\
\\
Opening a window for tolerating timing deviations allows an active attacker to inject a captured synchronization response package within that time window.
Nevertheless, because the header is protected by a \gls{mac2}, the only way to inject a synchronization package is to use exactly the same frame as captured,
i.e. the same source address, which will be discarded because the the requesting device already finished the synchronization stage.
\\
Additionally, the counter value is of no use for the attacker, as described above.

\subsection{High Level Cryptography Library}

\subsubsection{OpenSSL}

\begin{itemize}
 \item install libssl, libssl-dev
\end{itemize}

\subsubsection{Crypto++}
\begin{itemize}
 \item install  libcrypto++9
\end{itemize}

